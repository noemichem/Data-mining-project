\chapter{Introduction}

\section{Purpose of the project}
The music industry plays a significant role in today’s globalized world.
It has been growing rapidly over the years. On the other hand, the music
industry faces the unique challenge of addressing divers audiences across
different countries. Each country has different tastes in music which are shaped
by lots of factors. For record labels, understanding these preferences is
crucial for successful music promotion and marketing. The ability to predict
the popularity of a song before it is released can provide a significant 
competitive privilege.\\
\\
This project aims to analyze the varying musical tastes across different countries by 
understanding their popular songs and what makes those songs popular. To be able to understand
which factors make the song more listened than the others in a specific country, the audio features 
such as danceability, key, loudness, mode, speechiness and acousticness of the songs will be analyzed.
By leveraging Data Mining and Machine Learning Techniques, the project will provide insights into
the factors determining the popularity of songs in different countries.\\


\section{Dataset}
The final dataset used in this project is obtained by combining two different and extensive datasets.
The first dataset is taken from the Spotify API. The dataset contains audio features of 160k+ songs. The audio features
that are used in this project are danceability, energy, key, loudness, mode, speechiness, acousticness, instrumentalness,
liveness, valence, tempo, id, duration_ms, time_signature.  These features are essential for analyzing the musical 
elements that contribute to a song's popularity in each country.\\
The second dataset is taken from Kaggle which is called Spotify charts. (https://www.kaggle.com/datasets/dhruvildave/spotify-charts/data)
This dataset contains information about the most popular songs in the countries. It is a complete dataset 
of all the "Top 200" and "Viral 50" charts published globally by Spotify. Spotify publishes a new chart
 every 2-3 days. This is Spotify's entire collection since January 1, 2017.
 The included key details are the song's name, artist, spotify id, rank, and the number of streams 
 in each country. This dataset is helpful to undertand how musical preferences change across different
 parts of the world. \\
 To create a unified dataset, first preprocessing is performed on both datasets, and final versions of two different 
 datasets are combined by matching spotify ids column. This ensured that the audio features from the Spotify API were 
 correctly associated with the corresponding songs and their popularity metrics in different countries.\\
 The final dataset comprises 195000 rows of data in total for 69 different countries. The combined
 dataset is approximately 30 MB in size, making it both manageable for analysis and rich enough to uncover 
 meaningful insights.\\