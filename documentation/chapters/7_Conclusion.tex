\chapter{Conclusion}

\subsection{Future Work}
For further developments of the project, it could be interesting to take into consideration also the sentiment analysis.
This can be done by studying the lyrics of the songs as it was done in the literature review. 
Finding out if a happy song is more keen to be popular than a sad song could be interesting.\\
Sentiment analysis could also be done on the reviews of the songs.
This could be done for example by analyzing what people write about a specific song or a specific artist on social media. Also the most important music magazines could be analyzed to see if the reviews of the songs are positive or negative and how this affects the popularity of the song.\\
All this information can be used to improve the model.\\
Another interesting thing to do is to improve the User Interface of the application. As of now the application allows the user to insert the region and the different values of the audio features. It could be improved by adding more features, like the possibility to search for a specific song and see its popularity, or the possibility to see the most popular songs of a specific artist.\\



\subsection{Conclusion}
In general, finding what really makes a song popular and make correct predictions is a very difficult task as the popularity of a song is influenced by many factors.\\ 
This project has shown that it is possible to predict the popularity of a song with a good accuracy, which could help a lot in the music industry, but it could be improved.\\
The algorithms used in this project are proved to be more accurate than the ones used in the state of the art. In the state of art the more accurate algorithm is the Logistic Regression, which has an accuracy of 0.52, while in this project it is the Random Forests with an accuracy of 0.83.\\
According to the results of the scientific paper, danceability is the most important feature for a song to be popular. In our project, danceability is surely an important feature, but it is not the most important one.\\
Furthermore, we found out that some features are similar in different regions of the world, like Acousticness, which is important in all the regions, while some others are different, like Valence, which is important in some regions and not in others.\\
This shows that globalization has an influence on the music industry, but there are still some cultural differences that have to be taken into consideration.\\



%Add an image??
